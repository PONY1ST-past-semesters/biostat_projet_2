\documentclass[../main.tex]{subfiles}

\begin{document}
\begin{CJK*}{UTF8}{gbsn}
\section*{Exercice 2c}

Décrivez dans vos mots les trois tests statistiques discutés en Sections 5.3.1, 5.3.2,
5.3.3 du livre "Applied Survival Analysis Using R" de Moore (2016) pour les coefficients dans le modèle de Cox. 
Comme dans la Section 5.3., dans un deuxième temps, discutez/comparez entre les trois tests leurs avantages et
inconvénients. Cette réponse devrait consister en environ $8$ à $12$ lignes max au total.

\smallskip
\paragraph{Solution}

Écrire $h_i(t_j)$ pour le hasard de patient $i$ au temps d'échec $t_j$.
Si l'hypothèse de proportionnalité est vraie,
on peut écrire $h_i(t_j) = h_0(t_j) \psi_i = h_0(t_j) e^{z_i \beta}$
où $h_0$ est le hasard de référence, $z_i \in \{0, 1\}$ selon 
si le patient $i$ est dans le groupe de traitement ou le groupe de contrôle
et $\beta$ est le paramètre à estimer. 
La vraisemblance partielle est définie par 
$L(\beta) = \prod_{k=1}^D \frac{\psi_i}{\sum_{l \in R_k} \psi_l} $
où $D$ est le nombre d'échecs et $R_k$ est l'ensemble des patients à risque au temps $t_k$.
On teste l'hypothèse nulle $H_0 : \beta = 0$.
Fixer un niveau $\alpha \in (0, \frac{1}{2})$.
Écrire $l = \ln (L)$, $S = l'$, $I = - l''$ et soit $\hat{\beta} \in \argmax_{\beta} L(\beta)$.
Le test de Wald rejette $H_0$ si $\abs{\hat{\beta} \sqrt{I(\hat{\beta})} } > z_{\frac{\alpha}{2}}$.
Le test du score rejette $H_0$ si $\abs{\frac{S(0)}{\sqrt{I(0)}}} > z_{\frac{\alpha}{2}}$.
Le test du ratio vraisemblance rejette $H_0$ si $\abs{2(l(\hat{\beta})-l(0))} > \chi^2_{\frac{\alpha}{2},1}$.
Si on ne peut pas obtenir $\hat{\beta}$, il faut utiliser le test du score.
Supposons à partir de maintenant qu'on a obtient $\hat{\beta}$.
Le résultat du test du ratio vraisemblance a un avantage par rapport aux deux autres tests
qu'il est invariant contre les transformations monotones.
Si l'invariance n'est pas nécessaire, alors le test de Wald est plus commun et plus facile à calculer. ////

\end{CJK*}
\end{document}