\documentclass[../main.tex]{subfiles}

\begin{document}
\begin{CJK*}{UTF8}{gbsn}
\section*{Exercice 3a}
Pour le jeu de données 
\texttt{datasurv.txt}, 
écrire $t_1 < \cdots < t_D$ pour tous les temps distincts d'échec, 
$n_i$ pour le nombre de sujets à risque au temps $t_i$
et $d_i$ pour le nombre d'échecs au temps $t_i$.
Comme les patients à risque inclut ceux qui ont eu un échec,
on a $n_i \geqslant d_i \geqslant 1$ pour tout $i$.
On définit $q_i = \frac{d_i}{n_i}$ et $S(t) = \prod_{t_i \leqslant t} (1 - q_i)$
pour tout $t \in \{1, \cdots, t_D\}$.
Produisez un tableau de valeurs de $t_i$, $n_i$, $d_i$, $q_i$ et $S(t_i)$.

\smallskip
\paragraph{Solution}

On produit le tableau automatiquement en R :

\begin{lstlisting}
library(survival)
donnee <- read.delim("datasurv.txt", header = TRUE)
attach(donnee)
max_jour <- 500
t <- seq(max_jour)
n <- rep(0, max_jour)
d <- rep(0, max_jour)
q <- rep(0, max_jour)
S <- rep(0, max_jour)
n[1] <- sum(time_days >= 1)
S[1] <- 1

for (i in 2:max_jour) {
  n[i] <- sum(time_days >= i)
  if (i %in% time_days) {
    d[i] <- sum(event[which(time_days == i)])
  } 
  q[i] <- d[i] / n[i]
  S[i] <- S[i-1] * (1 - q[i])
}

tableau_a_remettre <- data.frame(
  cbind(t[sort(time_days[event == 1])], 
        n[sort(time_days[event == 1])], 
        d[sort(time_days[event == 1])], 
        q[sort(time_days[event == 1])],
        1 - q[sort(time_days[event == 1])],
        S[sort(time_days[event == 1])]))
names(tableau_a_remettre) <- c("t", "n", "d", "q", "1-q", "S")
print(tableau_a_remettre)
\end{lstlisting}

Le résultat produit par l'ordinateur est :

\begin{center}
\begin{tabular}{|c|c|c|c|c|c|}
\hline
$t$ & $n$ & $d$ & $q$ & $1-q$ & $S$ \\ \hline
196 & 18 & 1 & 0.0556 & 0.9444 & 0.9444 \\ \hline
258 & 13 & 1 & 0.0769 & 0.9231 & 0.8718 \\ \hline
262 & 12 & 1 & 0.0833 & 0.9167 & 0.7991 \\ \hline
375 & 5 & 1 & 0.2000 & 0.8000 & 0.6393 \\ \hline
377 & 3 & 1 & 0.3333 & 0.6667 & 0.4262 \\ \hline
409 & 2 & 1 & 0.5000 & 0.5000 & 0.2131 \\ \hline
\end{tabular}
\end{center}

Le calcul est complet. ////

\end{CJK*}
\end{document}