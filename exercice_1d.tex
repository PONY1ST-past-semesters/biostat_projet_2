\documentclass[../main.tex]{subfiles}

\begin{document}
\begin{CJK*}{UTF8}{gbsn}

\section*{Exercice 1d}
Pour le groupe \texttt{trt = 0}, trouvez le temps médian de survie et
construisez un intervalle de confiance à $95\%$ pour le temps médian de survie.
    
\paragraph{Solution}

Le code R crée d'abord un sous-ensemble des données \texttt{subsetdata} 
en filtrant les patients avec \texttt{trt == 0} de l'ensemble de données 
\texttt{diabetic} à l'aide de la fonction \texttt{subset}.
Ensuite, nous avons utilisé la fonction \texttt{survfit} 
pour estimer la durée de survie médiane de ce sous-ensemble 
et l'intervalle de confiance à $95\%$ correspondant.

\begin{lstlisting}
    subset_data <- subset(diabetic,trt == 0)
    fit <- survfit(Surv(time,status) ~ 1,data=subset_data, conf.type="log-log")
    result.km<-fit
    print(result.km)
\end{lstlisting}
    
La réponse est :
    
\begin{lstlisting}
    Call: survfit(formula = Surv(time, status) ~ 1, data = subset_data, 
        conf.type = "log-log")
        
            n events median 0.95LCL 0.95UCL
    [1,] 197    101   43.7    31.6    59.8
\end{lstlisting}

Donc :

\begin{enumerate}
    \item Chez les patients n'ayant pas reçu de traitement (\texttt{trt = 0}), la durée médiane de survie est de $43,7$ jours.
    \item L'intervalle de confiance à $95\%$ a une limite inférieure de $31,6$ jours et une limite supérieure de $59,8$ jours. ////
\end{enumerate}

\end{CJK*}
\end{document}
