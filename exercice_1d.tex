\documentclass[../main.tex]{subfiles}

\begin{document}
\begin{CJK*}{UTF8}{gbsn}

\section*{Exercice 1d}
Pour le groupe \texttt{trt = 0}, trouvez le temps médian de survie et
construisez un intervalle de confiance à $95\%$ pour le temps médian de survie.
    
\paragraph{Solution}\
    
Le but de ce problème est de se concentrer sur le groupe de patients qui n'ont pas reçu de traitement au laser (i.e., trt = 0) et de trouver le temps de survie médian dans ce sous-ensemble, qui est défini ici comme le temps écoulé entre le début du traitement et la cécité. Nous devons ensuite calculer un intervalle de confiance à 95\% pour ce temps de survie médian, et nous avons choisi d'utiliser la méthode "log-log" fournie par Barker (2009).
    
Le code R crée d'abord un sous-ensemble des données \texttt{subsetdata} en filtrant les patients avec \texttt{trt == 0}de l'ensemble de données \texttt{diabetic} à l'aide de la fonction subset. Ensuite, nous avons utilisé la fonction \texttt{survfit} pour estimer la durée de survie médiane de ce sous-ensemble et l'intervalle de confiance à 95\% correspondant, où nous avons choisi le type d'intervalle de confiance "log-log".
    
Selon les résultats obtenus :
\begin{enumerate}
    \item Chez les patients n'ayant pas reçu de traitement (\texttt{trt = 0}), la durée médiane de survie est de 43,7 jours.
    \item L'intervalle de confiance à 95\% a une limite inférieure de 31,6 jours et une limite supérieure de 59,8 jours.
\end{enumerate}
    
\begin{lstlisting}
subset_data <- subset(diabetic,trt == 0)
fit <- survfit(Surv(time,status) ~ 1,data=subset_data, conf.type="log-log")
result.km<-fit
print(result.km)
Call: survfit(formula = Surv(time, status) ~ 1, data = subset_data, 
    conf.type = "log-log")
    
        n events median 0.95LCL 0.95UCL
[1,] 197    101   43.7    31.6    59.8
\end{lstlisting}

\end{CJK*}
\end{document}
