\documentclass[../main.tex]{subfiles}

\begin{document}
\begin{CJK*}{UTF8}{gbsn}
\section*{Exercice 2ab}
Pour le jeu de données 
\texttt{rotterdam} dans la bibliothèque \texttt{survival} en R, 
on s'intéresse par l'association entre les variables explicatives \texttt{age}, \texttt{meno}, 
\texttt{sizes}, \texttt{grade}, \texttt{notes} et \texttt{chemo},
et la variable de réponse le temps jusqu'à une récidive du cancer au sein.
Ajustez un modèle de Cox pour cela et rapporter 
les rapports de risque pour toutes les variables explicatives avec leurs intervalles de confiance (à $95\%$ ?).
Nommez l'hypothèse importante du modèle de Cox. 
Expliquez sans le faire comment on peut vérifier cette hypothèse.

\paragraph{Solution}

à faire
\end{CJK*}
\end{document}
