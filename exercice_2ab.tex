\documentclass[../main.tex]{subfiles}

\begin{document}
\begin{CJK*}{UTF8}{gbsn}
\section*{Exercice 2ab}
Pour le jeu de données 
\texttt{rotterdam} dans la bibliothèque \texttt{survival} en R, 
on s'intéresse par l'association entre les variables explicatives \texttt{age}, \texttt{meno}, 
\texttt{sizes}, \texttt{grade}, \texttt{notes} et \texttt{chemo},
et la variable de réponse le temps jusqu'à une récidive du cancer au sein.
Ajustez un modèle de Cox pour cela et rapporter 
les rapports de risque pour toutes les variables explicatives avec leurs intervalles de confiance (à $95\%$ ?).
Nommez l'hypothèse importante du modèle de Cox. 
Expliquez sans le faire comment on peut vérifier cette hypothèse.

\paragraph{Solution}
Tout d'abord, nous importons l'ensemble de données et examinons les informations de base sur les données.

\begin{lstlisting}
library(rsq)

data("hcrabs")
attach (hcrabs)

\end{lstlisting}

Ensuite,nous réalisons 
un modèle de régression de Poisson avec la variable de réponse \texttt{num.satellites} 
et les variables explicatives \texttt{color}, \texttt{spine}, et \texttt{width}.

\begin{lstlisting}[language=R]
modele_poisson <- 
glm ( num.satellites ~ width + spine + color, family = poisson ( link = log ) ,
data = hcrabs ) 
\end{lstlisting}

On fait puis le test de surdispersion et évaluer le paramèter de sur-dispersion.

\begin{lstlisting}
library (AER)
print(dispersiontest(modele_poisson))
\end{lstlisting}

Le test donne que le paramètre de dispersion est $3.143975$ avec p-valeur $4.07e-08$.
Par conséquent, la sur-dispersion est significative.
Alors la dispersion observée dans les données 
dépasse ce qui serait attendu sous une distribution de Poisson. 
Dans notre cas, les variations du nombre de satellites mâles ne sont pas 
bien prises en compte dans le modèle de Poisson standard et 
des modèles de surdiscrétisation plus complexes peuvent être envisagés pour mieux s'adapter à ces données.
Lorsque nous effectuons une régression de Poisson, nous supposons que la variance est égale à la moyenne, 
une propriété inhérente à la distribution de Poisson. 
Cependant, les données réelles peuvent ne pas toujours respecter cette supposition. 

Comme on a une surdispersion, il faudrait avoir 
une hétérogénéité non observée ou une accumulation de comptages issus de plusieurs processus indépendants.
Face à la surdispersion, 
on se tourne généralement vers d'autres modèles plus complexes pour les données de comptage. 
Une choix est d'utilier un modèle mixte, qui s'appelle modèle binomial négatif. 
Il s'agit de commencer avec un modèle de Poisson
avec le paramètre estimé par une distribution de Gamma.

Passons au sujet prochain, il y a $2 \times 4-1=15$ modèles possibles si on ne compte pas l'intercept.
On choisit le critère AIC pour sélectionner le meilleur modèle.
On rappelle que, si on a $k$ modèles qui ont les valeurs d'AIC $\text{AIC}_1$, $\cdots$, $\text{AIC}_k$ respectivement,
les valeurs $\exp{-\frac{1}{2}(\text{AIC}_i - \text{AIC}_{min})}$ sont les probabilités que le modèle $i$
minimisant la perte d'informations.

\begin {lstlisting}[language=R]
modele_color <- glm ( num.satellites ~ color, family = poisson ( link = log ) ,data = hcrabs ) 
modele_spine <- glm ( num.satellites ~ spine, family = poisson ( link = log ) ,data = hcrabs )
modele_width <- glm ( num.satellites ~ width, family = poisson ( link = log ) ,data = hcrabs )
modele_weight <- glm ( num.satellites ~ weight, family = poisson ( link = log ) ,data = hcrabs )
modele_color_spine <- glm ( num.satellites ~ color + spine, family = poisson ( link = log ) ,data = hcrabs )
modele_color_width <- glm ( num.satellites ~ color + width, family = poisson ( link = log ) ,data = hcrabs )
modele_color_weight <- glm ( num.satellites ~ color + weight, family = poisson ( link = log ) ,data = hcrabs )
modele_spine_width <- glm ( num.satellites ~ spine + width, family = poisson ( link = log ) ,data = hcrabs )
modele_spine_weight <- glm ( num.satellites ~ spine + weight, family = poisson ( link = log ) ,data = hcrabs )
modele_width_weight <- glm ( num.satellites ~ width + weight, family = poisson ( link = log ) ,data = hcrabs )
modele_color_spine_width <- glm ( num.satellites ~ color + spine + width, family = poisson ( link = log ) ,data = hcrabs )
modele_color_spine_weight <- glm ( num.satellites ~ color + spine + weight, family = poisson ( link = log ) ,data = hcrabs )
modele_color_width_weight <- glm ( num.satellites ~ color + width + weight, family = poisson ( link = log ) ,data = hcrabs )
modele_spine_width_weight <- glm ( num.satellites ~ spine + width + weight, family = poisson ( link = log ) ,data = hcrabs )
modele_color_spine_width_weight <- glm ( num.satellites ~ color + spine + width + weight, family = poisson ( link = log ) ,data = hcrabs )

modeles <- list(modele_color, modele_spine, modele_width, modele_weight, 
modele_color_spine, modele_color_width, modele_color_weight, modele_spine_width, 
modele_spine_weight, modele_width_weight, modele_color_spine_width, 
modele_color_spine_weight, modele_color_width_weight, 
modele_spine_width_weight, modele_color_spine_width_weight)

AICs <- rep(0, 15)
for (i in 1:15) {
  AICs[i] <- AIC(modeles[[i]])
}

print(AICs)

min_AIC <- min(AICs)
proba <- rep(0, 15)
for (i in 1:15) {
  proba[i] <- exp(0.5*(min_AIC- AICs[i]))
}

print(proba)

print(which.max(proba))

\end{lstlisting}

À la fin, on trouve que le modèle color-weight est le meilleur modèle.
Pour le résidus, on choisit les résidus d'Anscombe. 
Les résidus d'Anscombe sont spécifiquement conçus pour 
les modèles linéaires généralisés et 
offrent de bonnes propriétés pour identifier des valeurs atypiques ou des observations influentes. 
Ce lien \url{https://www.sfu.ca/sasdoc/sashtml/insight/chap39/sect57.htm} 
fournit des informations supplémentaires sur les résidus d'Anscombe.

\begin {lstlisting}[language=R]
  library(surveillance)
  plot(anscombe.residuals(modele_color_weight, phi =1))
\end{lstlisting}

Toutes les modélisations sont complètes.////
\end{CJK*}
\end{document}
