\documentclass[../main.tex]{subfiles}

\begin{document}
\begin{CJK*}{UTF8}{gbsn}
\section*{Exercice 3}
On s’intéresse à l’ensemble de variables suivantes: \texttt{age}, \texttt{sex}, \texttt{SES}, \texttt{smoking} et
\texttt{weight}, notant que les variables \texttt{ID} et \texttt{time discrete} représentent respectivement l’identifiant
de l’individu et le temps (il y a quatre temps: vague 1, 2, 3 ou 4) dans le jeu de données \texttt{addhealth long}.

Proposez une structure de corrélations entre les mesures de poids de chaque individu.
Ensuite, réalisez et comparez les trois modèles suivants utilisant la corrélation résiduelle :

\begin{enumerate}
    \item Un modèle de régression linéaire pour la variable dépendente \texttt{weight} sans aucune variable explicative,
    c'est-à-dire just une ordonnée à l'origine.
    \item Un modèle de régression linéaire pour la variable dépendente \texttt{weight} et les variables explicatives
    \texttt{age} et \texttt{sex}.
    \item Un modèle de régression linéaire pour la variable dépendente \texttt{weight} et les variables explicatives
    \texttt{age}, \texttt{sex}, \texttt{SES} et \texttt{smoking}.
\end{enumerate}

Finalement, afin d'étudier la rélation entre la variable dépendent \texttt{weight} et la variable explicative \texttt{SES},
on propose quatre modèles suivants:

\begin{enumerate}
    \item Un modèle linéaire en assumant l'indépendance entre toutes les observations.
    \item Un modèle linéaire généralisé (normale) où on assume une matrice de corrélation interchangeable.
    \item GEE avec une matrice de corrélation interchangeable.
    \item Un modèle mixte avec une ordonnée à l'origine et une pente aléatoire pour chaque individu.
\end{enumerate}

Comparez les quatre modèles utilisant la variance de la coefficient de la variable continue \texttt{SES},
pour laquelle faisez un intervalle de confiance.

\paragraph{Solution}

Parmi les quatres stuctures mentionnés en classe, on décide que la matrice de corrélation 
est celui de Toepliz. On ne choisit pas les trois autres parce que les poids d'un individu 
entre les points différents de temps ne sont pas indépendents ou interchangeable et que l'on 
croit pas les poids forment une chaîne de Markov. On choisit Toepliz parce qu'il y a une « délai »
entre les observations et que nous n'avons pas trouvé une autre matrice de corrélation en ligne 
qui semble plus appropriée.

D'abord, on réalise les trois modèles linéaires :

\begin{enumerate}
    \item Le modèle avec seulement une ordonnée à l'origine :

\begin{lstlisting}
    modeles_lineaires_aucune <- list()
    residues_aucune <- rep(1,4)
    for (i in 1:4){
        modele_i <- lm(donnee_separe[[i]]$weight ~ 1)
        modeles_lineaires_aucune <- append(modeles_lineaires_aucune, modele_i)
        residues_aucune[i] <- mean(residuals(modele_i))
    }
\end{lstlisting}

    \item Le modèle avec \texttt{age} et \texttt{sex} comme variables explicatives :

\begin{lstlisting}
    modeles_lineaires_age_sex <- list()
    residues_age_sex <- rep(1,4)
    for (i in 1:4){
        modele_i <- lm(donnee_separe[[i]]$weight ~ donnee_separe[[i]]$age + donnee_separe[[i]]$sex)
        modeles_lineaires_age_sex <- append(modeles_lineaires_age_sex, modele_i)
        residues_age_sex[i] <- mean(residuals(modele_i))
    }
\end{lstlisting}

    \item Le modèle avec \texttt{age}, \texttt{sex}, \texttt{SES} et \texttt{smoking} comme variables explicatives : 

\begin{lstlisting}
    modeles_lineaires_age_SES_sex_smoking <- list()
    residues_age_SES_sex_smoking <- rep(1,4)
    for (i in 1:4){
        modele_i <- lm(donnee_separe[[i]]$weight ~ donnee_separe[[i]]$age + donnee_separe[[i]]$sex + 
    donnee_separe[[i]]$SES + donnee_separe[[i]]$smoking)
        modeles_lineaires_age_SES_sex_smoking <- append(
    modeles_lineaires_age_SES_sex_smoking,  modele_i)
        residues_age_SES_sex_smoking[i] <- mean(residuals(modele_i))
    }
    \end{lstlisting}   
\end{enumerate}

On imprime puis les corrélations résiduelles :

\begin{lstlisting}
    print("Les correlations residuelles sont :")
    print(cor(residues_aucune, residues_age_sex)) # -0.5644551
    print(cor(residues_age_sex, residues_age_SES_sex_smoking)) # -0.9982123
    print(cor(residues_age_SES_sex_smoking, residues_aucune)) # 0.5159088
\end{lstlisting}

\begin{itemize}
    \item \(-0.5644551\): 
Cette valeur représente la corrélation entre les résidus du premier modèle  et ceux du deuxième modèle 
(avec \textit{age} et \textit{sex} comme variables explicatives). 
Cette corrélation négative signifie que lorsque les résidus du premier modèle augmentent, 
ceux du deuxième modèle ont tendance à diminuer, et vice versa.
    
    \item \(-0.9982123\): 
Cette valeur représente la corrélation entre les résidus du deuxième modèle et 
ceux du troisième modèle. Cette corrélation, très proche de $-1$, 
indique une forte relation négative entre les résidus des deux modèles. 
Cela signifie que lorsque les résidus du deuxième modèle augmentent, 
ceux du troisième modèle ont tendance à diminuer, et vice versa.
    
    \item \(0.5159088\): 
Cette valeur représente la corrélation entre les résidus du troisième modèle et ceux du premier modèle. 
Cette corrélation positive signifie que lorsque les résidus du troisième modèle augmentent, 
ceux du premier modèle ont également tendance à augmenter, et vice versa.
\end{itemize}
   
À partir de maintenant, on utilise \texttt{SES} comme une variable explicative pour la variable de réponse \texttt{weight}.
Les modèles démandés sont réalisés ci-dessous :

\begin{enumerate}
    \item Modèle linéaire en assumant l'indépendance entre toutes les observations:

\begin{lstlisting}
    modele_lineaire <- lm(weight ~ SES, data = donnee)
    print(summary(modele_lineaire))  
\end{lstlisting}

On trouve que l'écat type d'erreur pour \texttt{SES} est $0.1328$.
    \item Modèle linéaire généralisé (normale) où on assume une matrice de corrélation interchangeable:

\begin{lstlisting}
    library(nlme)
    glm_normal <- gls(weight ~ SES, data = donnee, correlation = corCompSymm(form = ~ 1 | SES))
    print(summary(glm_normal)) 
\end{lstlisting}

On trouve que l'écat type d'erreur pour \texttt{SES} est $0.1868304$.
    \item GEE avec une matrice de corrélation interchangeable:
    
\begin{lstlisting}
    require(geepack)
    GEE <- summary(geese(weight ~ SES, id = ID, data = donnee, corstr = 'exchangeable'))
    print(GEE) 
\end{lstlisting}
On trouve que l'écat type d'erreur pour \texttt{SES} est $0.1646921$.

    \item Modèle mixte avec une ordonnée à l'origine et une pente aléatoire pour chaque individu:
\begin{lstlisting}
    library(lme4)
    modele_mixte <- lmer(weight ~ (1 + SES | ID), data = donnee)
    print(summary(modele_mixte)) 
    print(confint(modele_lineaire))
\end{lstlisting}
Nous avons manipulé beaucoup d'options. Ce modèle ne converge jamais. 
\end{enumerate}

On sait que des erreurs standard plus petites impliquent 
des estimations plus précises des paramètres du modèle, 
tandis que des erreurs standard plus grandes indiquent une plus grande incertitude. 
Ainsi, parmi les quatre modèles mentionnés ci-dessus, 
le permier modèle présente l'erreur standard la plus faible ($0.1328$).
Par conséquent, le premier modèle est le meilleur modèle pour les données.
On puis construire le $95 \%$ intervalle de confiance :

\begin{lstlisting}
    print(confint(modele_lineaire))
\end{lstlisting}

L'intervalle pour l'ordonnée à l'origine est $[161.6002042, 165.2151458]$
et l'intervalle pour la pente est $[-0.6829711, -0.1625338]$.
Donc, en général, on peut dire que le poids diminue lorsque \texttt{SES} augmente. ////

\end{CJK*}
\end{document}


