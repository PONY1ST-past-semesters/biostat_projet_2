\documentclass[../main.tex]{subfiles}

\begin{document}
\begin{CJK*}{UTF8}{gbsn}
    
\section*{Exercice 2e}
Soit $\{X^{(i)} = (X^{(i)}_1,X^{(i)}_2)\}_{i=1}^n$ et $\{Y_i\}_{i=1}^n$
deux suites de variables aléatoires indépendantes et identiquement distribuées
à valeur $\mathbb{R}^2$ et $\mathbb{R}$ respectivement définies 
sur un espace de probabilité commun $(\Omega, \mathcal{F}, P)$ tels qu'il 
existe trois constantes $\beta_0$, $\beta_1$ et $\beta_2$ réels avec, 
pour chaque $i \in \{1, \cdots, n\}$ :

\begin{equation*}
    \mathbb{E} \squared{Y_i \mid X^{(i)}} = \beta_0 + \beta_1 X^{(i)}_1 + \beta_2 X^{(i)}_2
\end{equation*}

Avec les hypothèses appropriées, postulez une équation d'estimation pour $\beta_0$, $\beta_1$ et $\beta_2$.
Calculez une expression pour $\beta_0$, $\beta_1$ et $\beta_2$ et ses variances étant donné les $X$.
Réalisez tout utilisant le donnée \texttt{addhealth} avec $Y_i = \texttt{Weight}_i$,
$X^{(i)}_1 = \texttt{age}_i$ et $X^{(i)}_2 = \texttt{SES}_i$.

\smallskip
\paragraph{Solution}
On fait les hypothèses suivantes:

\begin{enumerate}
    \item La règle de décision est moindre carrés.
    \item Seulement pour le calcul des variances, on suppose que $Y_1$ a distribution $N(0, \sigma^2)$.
\end{enumerate}

Pour simplicité, écrire :

\begin{equation*}
    Y = 
    \begin{bmatrix}
        Y_1 \\
        \vdots \\
        Y_n
    \end{bmatrix}; \tab[0.5cm]
    X =
    \begin{bmatrix}
        1 & X^{(1)}_1 & X^{(1)}_2 \\
        \vdots & \vdots & \vdots \\
        1 & X^{(n)}_1 & X^{(n)}_2
    \end{bmatrix}; \tab[0.5cm]
    \beta =
    \begin{bmatrix}
        \beta_0 \\
        \beta_1 \\
        \beta_2
    \end{bmatrix}; \tab[0.5cm]
    S(X,Y,\beta) = (Y-X\beta)^T(Y-X\beta)
\end{equation*}

L'espace de paramètre est $\mathbb{R}^3$.
Selon la définition d'éaquation d'estimation, il s'agit de trouver, pour chaque $i \in \{1, \cdots, n\}$,
une fonction $\psi_i : \mathbb{R}^3 \times \mathbb{R}^3 \to \mathbb{R}^3$ telle que, pour 
tout $\beta \in \mathbb{R}^3$, on a :

\begin{equation*}
    \mathbb{E} \squared{ \sum_{i=1}^n \psi_i((Y_i, X^{(i)}_1, X^{(i)}_2), \beta)} = 
    \begin{bmatrix}
        0 \\
        0 \\
        0 
    \end{bmatrix}
\end{equation*}

On écrit $\hat{\beta}$ pour la solution de l'équation d'estimation, 
qui est une variable aléatoire dépendant $Y$ et $X$, telle que 
$\sum_{i=1}^n \psi_i((Y_i, X^{(i)}_1, X^{(i)}_2), \hat{\beta}) = 0$ presque partout.
Selon la première hypothèse, la règle de décision est :

\begin{equation*}
    \hat{\beta} \in \argmin_{\beta \in \mathbb{R}^3} S(X,Y,\beta)
\end{equation*}

Donc on utilise :

\begin{equation*}
    \begin{split}
    &\tab[0.4cm] \frac{\partial}{\partial \beta }S(X,Y,\beta) = -2X^T(Y-X\beta) = 
    \sum_{i=1}^n
    \begin{bmatrix}
        (-2Y_i + 2\beta_0 + 2\beta_1 X^{(i)}_1 + 2\beta_2 X^{(i)}_2) \\
        (-2Y_i X^{(i)}_1 + 2\beta_0 X^{(i)}_1 + 2\beta_1 (X^{(i)}_1)^2 + 2\beta_2 X^{(i)}_1 X^{(i)}_2) \\
        (-2Y_i X^{(i)}_2 + 2\beta_0 X^{(i)}_2 + 2\beta_1 X^{(i)}_1 X^{(i)}_2 + 2\beta_2 (X^{(i)}_2)^2)
    \end{bmatrix} \\ & =
    \sum_{i=1}^n \psi_i((Y_i, X^{(i)}_1, X^{(i)}_2), \beta) =
    \begin{bmatrix}
        0 \\
        0 \\
        0
    \end{bmatrix}
    \end{split}
\end{equation*}

Comme :

\begin{equation*}
    \begin{split}
    & \tab[0.4cm] \mathbb{E} [\psi_i((Y_i, X^{(i)}_1, X^{(i)}_2), \beta)] = 
    \begin{bmatrix}
        \mathbb{E} \squared{(-2Y_i + 2\beta_0 + 2\beta_1 X^{(i)}_1 + 2\beta_2 X^{(i)}_2)} \\
        \mathbb{E} \squared{(-2Y_i X^{(i)}_1 + 2\beta_0 X^{(i)}_1 + 2\beta_1 (X^{(i)}_1)^2 + 2\beta_2 X^{(i)}_1 X^{(i)}_2)} \\
        \mathbb{E} \squared{(-2Y_i X^{(i)}_2 + 2\beta_0 X^{(i)}_2 + 2\beta_1 X^{(i)}_1 X^{(i)}_2 + 2\beta_2 (X^{(i)}_2)^2)}
    \end{bmatrix} \\ & =
    \begin{bmatrix}
        \mathbb{E} \squared{\mathbb{E} \squared{(-2Y_i + 2\beta_0 + 2\beta_1 X^{(i)}_1 + 2\beta_2 X^{(i)}_2) \mid X^{(i)}}} \\
        \mathbb{E} \squared{X^{(i)}_1\mathbb{E} \squared{-2Y_i + 2\beta_0 + 2\beta_1 X^{(i)}_1 + 2\beta_2 X^{(i)}_2 \mid X^{(i)}}} \\
        \mathbb{E} \squared{X^{(i)}_2\mathbb{E} \squared{-2Y_i + 2\beta_0 + 2\beta_1 X^{(i)}_1 + 2\beta_2 X^{(i)}_2 \mid X^{(i)}}}
    \end{bmatrix} =
    \begin{bmatrix}
        0 \\
        0 \\
        0
    \end{bmatrix}
    \end{split}
\end{equation*}

On conclut que l'équation définie est une équation d'estimation. On a donc :

\begin{equation*}
    \hat{\beta} = (X^TX)^{-1}X^TY
\end{equation*}

L'inverse peut être un inverse généralisé si $X^TX$ n'est pas inversible.
Selon la deuxième hypothèse, on sait immédiatement que $\text{Var}(\hat{\beta} \mid X) = \sigma^2 (X^TX)^{-1}$.
Maintenant, il s'agit de réaliser tout utilisant le donnée \texttt{addhealth} avec $Y_i = \texttt{Weight}_i$,
$X^{(i)}_1 = \texttt{age}_i$ et $X^{(i)}_2 = \texttt{SES}_i$.
Ce travail ne consiste que l'utilisation de la fonction \texttt{lm} en R :

\begin{lstlisting}
    set.seed(1234)

    donnee <- read.delim("Chapters\\biostat_projet_1\\resource_content_1_addhealth.txt", header = TRUE)
    #donnee <- read.delim("resource_content_1_addhealth.txt", header = TRUE)
    
    attach(donnee)
    
    #On commence par le netoyage
    donnee$feeling_depressed <- as.factor(donnee$feeling_depressed)
    donnee$feeling_depressed[is.na(donnee$feeling_depressed)] <- as.factor(
        floor(runif(sum(is.na(donnee$feeling_depressed)), min = 1, max = 4.9999)))
    
    donnee$smoking <- as.factor(donnee$smoking)
    
    donnee$weight <- as.numeric(donnee$weight)
    donnee$weight[is.na(donnee$weight)] <- mean(donnee$weight, na.rm = TRUE)
    
    donnee$time <- as.factor(donnee$time) #identiquement 1
    
    donnee$age <- as.numeric(donnee$age)
    donnee$age[is.na(donnee$age)] <- as.factor(mean(donnee$age, na.rm = TRUE))
    
    donnee$sex <- as.factor(donnee$sex)
    donnee$sex[is.na(donnee$sex)] <- as.factor(floor(runif(sum(is.na(donnee$sex)), min = 1, max = 2.9999)))
    
    donnee$SES <- as.numeric(donnee$SES)
    donnee$SES[is.na(donnee$SES)] <- as.numeric(floor(mean(donnee$SES, na.rm = TRUE)))
    
    attach(donnee)
    
    modele_moindre_carre <- lm(weight ~ SES + age, data = donnee)
    print(summary(modele_moindre_carre))
\end{lstlisting}

Le code donne que, selon les donné, on a $\beta_0 = 59.5309$, $\beta_1 = 5.5819$ et $\beta_2 = -0.3637$.
Les variances sont respectivement $3.7017^2 = 13.70258289 $, $0.2223^2 = 0.04941729$ et $0.2149^2 = 0.4618201$.
Le calcul et la démonstration sont complets. ////
\end{CJK*}
\end{document}