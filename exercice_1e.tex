\documentclass[../main.tex]{subfiles}

\begin{document}
\begin{CJK*}{UTF8}{gbsn}

\section*{Exercice 1e}
Faites un test de Log-Rank pour comparer les deux groupes \texttt{trt = 0} et \texttt{trt = 1}. 
Discutez les résultats.
    
\paragraph{Solution}
On effectue un test de Log-Rank :
    
\begin{lstlisting}
result.logrank <- survdiff(Surv(time, status) ~ trt, data = diabetic)
print(result.logrank)
\end{lstlisting}

La réponse est :

\begin{lstlisting}
Call:
survdiff(formula = Surv(time, status) ~ trt, data = diabetic)
    
        N Observed Expected (O-E)^2/E
trt=0 197      101     71.8      11.9
trt=1 197       54     83.2      10.3
        (O-E)^2/V
trt=0      22.2
trt=1      22.2
    
Chisq= 22.2  on 1 degrees of freedom, p= 2e-06 
\end{lstlisting}

Comme la p-valeur est $2 \times 10^{-6}$, qui est bien inférieure à $0,05$,
la différence de temps avant la cécité entre les deux 
groupes de traitement est statistiquement significative. 
Dans ce cas, le rapport de risque entre le groupe \texttt{trt=1}
et le groupe \texttt{trt=0} est plus petit que $1$.
Il y a donc suffisamment de preuve statistique pour dire que 
le traitement au laser peut diminuer le risque de cécité.
Cela implique que le 
traitement au laser peut être une intervention 
efficace pour retarder la perte de vision chez les patients diabétiques. ////

\end{CJK*}
\end{document}
