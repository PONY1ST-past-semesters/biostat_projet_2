\documentclass[../main.tex]{subfiles}

\begin{document}
\begin{CJK*}{UTF8}{gbsn}

\section*{Exercice 1e}
Faites un test de Log-Rank pour comparer les deux groupes. Discutez les résultats.
    
\paragraph{Solution}
On effectue un test de Log-Rank :
    
\begin{lstlisting}
result.logrank <- survdiff(Surv(time, status) ~ trt, data = diabetic)
print(result.logrank)
\end{lstlisting}

La réponse est :

\begin{lstlisting}
Call:
survdiff(formula = Surv(time, status) ~ trt, data = diabetic)
    
        N Observed Expected (O-E)^2/E
trt=0 197      101     71.8      11.9
trt=1 197       54     83.2      10.3
        (O-E)^2/V
trt=0      22.2
trt=1      22.2
    
Chisq= 22.2  on 1 degrees of freedom, p= 2e-06 
\end{lstlisting}

Comme la p-valeur est $2e-06$, qui est bien inférieure à $0,05$,
 la différence de temps avant la cécité entre les deux 
groupes de traitement est statistiquement significative. 
En outre, dans ce cas, le nombre de cécités était significativement 
plus élevé que prévu dans le groupe qui n'a pas reçu de traitement au laser 
(\texttt{trt=0}), tandis que le nombre de cécités était plus faible que prévu 
dans le groupe qui a reçu un traitement au laser (\texttt{trt=1}).
    
En conclusion, les résultats du test Log-Rank ont 
montré que le fait de recevoir ou non un traitement 
au laser a un effet significatif sur la courbe 
de survie des patients. Cela implique que le 
traitement au laser peut être une intervention 
efficace pour retarder la perte de vision chez les patients diabétiques. ////

\end{CJK*}
\end{document}
