\documentclass[../main.tex]{subfiles}

\begin{document}
\begin{CJK*}{UTF8}{gbsn}

\section*{Exercice 1f}
Finalement, on se demande si d'autres variables
dans le jeu de données pourraient être des facteurs confondants, et si l'on devrait stratifier le
test du Log-Rang sur l'une de ces variables. À partir de statistiques descriptives, de graphiques
et/ou d'arguments adaptés au contexte de l'étude, discuter de laquelle des variables du jeu de
données risque d'agir comme facteur confondant et reproduire le test du Log-Rang stratifié pour cette variable.
    
\paragraph{Solution}

%???
Dans notre analyse approfondie de l'ensemble de données sur les diabetic, 
nous avons adopté une stratégie basée sur la segmentation de l'âge médian 
afin de discerner l'effet des différents traitements sur les patients d'âges différents. 
Cette segmentation a permis de classer les patients en deux groupes, "Younger" et "Older", 
ce qui nous a permis d'explorer l'impact potentiel du facteur âge sur les résultats du traitement.
    
On a ensuite tracé les courbes de survie à l'aide de la fonction plot, 
en utilisant le rouge pour représenter le groupe des "Younger" 
et le bleu pour représenter le groupe des "Older". 
Le graphique montre que la courbe rouge est plus élevée que la courbe bleue 
à la plupart des moments, ce qui signifie que la probabilité de survie est plus 
élevée pour le groupe des "Younger" que pour le groupe des "Older". 
    
On a ensuite effectué des tests de Log-Rank 
stratifiés pour comparer les différences de durée de survie 
entre les groupes de traitement après prise en compte des facteurs liés à l'âge.
Les résultats ont montré une valeur p extrêmement faible (1e-06) 
indiquant une différence significative dans le délai de cécité entre les deux groupes de traitement.
Même après stratification par âge, il existe toujours une différence 
significative dans le risque de cécité entre le groupe de traitement $1$
et le groupe de \texttt{trt=0}. Nous pouvons conclure que les patients 
du groupe de \texttt{trt=1} ont une probabilité de cécité plus faible que 
les patients du groupe de traitement $0$. 
En outre, le facteur âge joue un rôle dans cette différence : 
le groupe de patients "Younger"
présente une probabilité de cécité plus faible que le groupe de patients "Older". 
Cela suggère que l'âge est un facteur important influençant la durée 
de la cécité et qu'il doit être pris en compte lors de l'élaboration des stratégies de traitement.

\begin{lstlisting}
median_age <- median(diabetic$age, na.rm = TRUE)
diabetic$age_group <- ifelse(diabetic$age <= median_age, "younger", "older")
result.kmage <- survfit(Surv(time, status) ~ age_group, data = diabetic)
plot(result.kmage, main = 'Courbe de Kaplan-Meier', xlab = 'Temps en jours', ylab = 'Probabilite de survie', col = c("red", "blue"))
survdiff(Surv(time, status) ~ trt + strata(age), data = diabetic)
    
Call:
survdiff(formula = Surv(time, status) ~ trt + strata(age), data = diabetic)
    
        N Observed Expected (O-E)^2/E (O-E)^2/V
trt=0 197      101     71.9      11.8      23.4
trt=1 197       54     83.1      10.2      23.4
    
    Chisq= 23.4  on 1 degrees of freedom, p= 1e-06
\end{lstlisting}

\end{CJK*}
\end{document}