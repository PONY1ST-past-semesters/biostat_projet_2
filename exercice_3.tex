\documentclass[../main.tex]{subfiles}

\begin{document}
\begin{CJK*}{UTF8}{gbsn}
\section*{Exercice 3a}
Pour le jeu de données 
\texttt{datasurv.txt}, 
écrire $t_1 < \cdots < t_D$ pour tous les temps distincts d'échec, 
$n_i$ pour le nombre de sujets à risque au temps $t_i$
et $d_i$ pour le nombre d'échecs au temps $t_i$.
Comme les patients à risque inclut ceux qui ont eu un échec,
on a $n_i \geqslant d_i \geqslant 1$ pour tout $i$.
On définit $q_i = \frac{d_i}{n_i}$ et $S(t) = \prod_{t_i \leqslant t} (1 - q_i)$
pour tout $t \in \{1, \cdots, t_D\}$.
Produisez un tableau de valeurs de $t_i$, $n_i$, $d_i$, $q_i$ et $S(t_i)$.

\begin{table}[h]
\centering
\begin{tabular}{|c|c|c|c|c|c|}
\hline
t & n & d & q & 1-q & S \\ \hline
196 & 18 & 1 & 0.0556 & 0.9444 & 0.9444 \\ \hline
258 & 13 & 1 & 0.0769 & 0.9231 & 0.8718 \\ \hline
262 & 12 & 1 & 0.0833 & 0.9167 & 0.7991 \\ \hline
375 & 5 & 1 & 0.2000 & 0.8000 & 0.6393 \\ \hline
377 & 3 & 1 & 0.3333 & 0.6667 & 0.4262 \\ \hline
409 & 2 & 1 & 0.5000 & 0.5000 & 0.2131 \\ \hline
\end{tabular}
\caption{Résultats de l'analyse de survie}
\label{table:surv_analysis}
\end{table}

\begin{lstlisting}
library(survival)
donnee <- read.delim("Desktop/u de m/STT6510/projet2/datasurv.txt", header = TRUE, sep = " ")
#donnee <- read.delim("datasurv.txt", header = TRUE)
attach(donnee)

#3a
#surv_objet <- Surv(time_days, event)
#print(grepl("\\+$", surv_objet[1]))

max_jour <- 500
t <- seq(max_jour)
n <- rep(0, max_jour)
d <- rep(0, max_jour)
q <- rep(0, max_jour)
S <- rep(0, max_jour)
n[1] <- sum(time_days >= 1)
S[1] <- 1

for (i in 2:max_jour) {
  n[i] <- sum(time_days >= i)
  if (i %in% time_days) {
    d[i] <- sum(event[which(time_days == i)])
  } 
  q[i] <- d[i] / n[i]
  S[i] <- S[i-1] * (1 - q[i])
}

tableau_a_remettre <- data.frame(
  cbind(t[sort(time_days[event == 1])], 
        n[sort(time_days[event == 1])], 
        d[sort(time_days[event == 1])], 
        q[sort(time_days[event == 1])],
        1 - q[sort(time_days[event == 1])],
        S[sort(time_days[event == 1])]))
names(tableau_a_remettre) <- c("t", "n", "d", "q", "1-q", "S")
print(tableau_a_remettre)

t  n d          q       1-q         S
1 196 18 1 0.05555556 0.9444444 0.9444444
2 258 13 1 0.07692308 0.9230769 0.8717949
3 262 12 1 0.08333333 0.9166667 0.7991453
4 375  5 1 0.20000000 0.8000000 0.6393162
5 377  3 1 0.33333333 0.6666667 0.4262108
6 409  2 1 0.50000000 0.5000000 0.2131054

\end{lstlisting}

\section*{Exercice 3bc}
En R, réalisez un graphique de $q$ en fonction de $t$ pour $t \in \{1, \cdots, 500\}$.
Ici, $q(t) = q_i$ si $t \in \{1, \cdots, t_D\}$ et $q(t) = 0$.Sinon,
Produisez en particulière le même graphique pour $t \in \{200, \cdots, 260\}$.

\begin{lstlisting}
library(ggplot2)

p <- ggplot(data.frame(t = t, q = q), aes(x = t, y = q)) +
  geom_line() +  labs(title = "Line Chart Example", x = "Time", y = "Value")
show(p)

\end{lstlisting}

Finalement, lissez cela utilisant le noyau d'Epanechnikov avec un paramètre de lissage $5$.
Plus précisément, on définit le noyau d'Epanechnikov comme $K(u) = \frac{3}{4}(1-u^2)\chi_{[-1,1]}(u)$
pour tout $u \in \mathbb{R}$ et on définit la fonction de lissage comme :

\begin{equation*}
    \hat{h}(t) = \frac{1}{b} \sum_{i=1}^D K\bracket{\frac{t-t_i}{b}} q_i
\end{equation*}

Ici $b = 5$. 
Produisez un graphique de $\hat{h}$ en fonction de $t$ pour $t \in \{200, \cdots, 260\}$.
Vérifiez les résultats en main pour $t = \{200, 210, 220, 230, 240, 250\}$.

\section*{Exercice 3d}

\begin{lstlisting}
#3d
debut <- 200
fin <- 260

p <- ggplot(data.frame(t = t[debut:fin], q = q[debut:fin]), aes(x = t, y = q)) +
  geom_line() +  labs(title = "Line Chart Example", x = "Time", y = "Value")
show(p)

epanechnikov <- function(u) {
  if (abs(u) <= 1) {
    return(0.75 * (1 - u^2))
  } else {
    return(0)
  }
}

b <- 5
D <- length(t[sort(time_days[event == 1])])
lisse <- rep(0, fin - debut)
for (i in debut:fin) {
  diff_seq <- (i - t[sort(time_days[event == 1])])/b
  epan <- rep(0, D)
  for (j in 1:D) {
    epan[j] <- epanechnikov(diff_seq[j])
  }
  lisse[i - debut + 1] <- (1/b)*sum(q[sort(time_days[event == 1])] * epan)
}

p <- ggplot(data.frame(t = t[debut:fin], lisse = lisse), aes(x = t, y = lisse)) +
  geom_line() +  labs(title = "Line Chart Example", x = "Time", y = "Value")
show(p)

\end{lstlisting}


à complet

\end{CJK*}
\end{document}
