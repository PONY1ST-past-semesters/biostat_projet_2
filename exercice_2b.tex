\documentclass[../main.tex]{subfiles}

\begin{document}
\begin{CJK*}{UTF8}{gbsn}
    
\section*{Exercice 2b}
Nommez l'hypothèse importante du modèle de Cox. 
Expliquez sans le faire comment on peut vérifier cette hypothèse.
    
\paragraph{Solution}

%????
$H_0:$ Les rapports de risque de toutes les variables explicatives 
sont indépendants du temps, signifiant que l'hypothèse des risques proportionnels est valide.
$H_1:$ Le rapport de risque d'au moins une variable 
explicative varie avec le temps, indiquant que l'hypothèse des risques proportionnels n'est pas valide.
    
Nous pouvons utiliser les diagrammes des résidus de Schoenfeld, 
qui montrent les résidus de Schoenfeld 
(une mesure de l'impact des variables sur le risque au fil du temps) 
en fonction du temps. Si l'hypothèse de risque proportionnel se vérifie, 
ces points ne présenteraient aucune tendance systématique dans le temps, 
c'est-à-dire qu'ils seraient distribués de manière aléatoire autour d'une ligne horizontale.

\end{CJK*}
\end{document}