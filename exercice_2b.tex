\documentclass[../main.tex]{subfiles}

\begin{document}
\begin{CJK*}{UTF8}{gbsn}
    
\section*{Exercice 2b}
Nommez l'hypothèse importante du modèle de Cox. 
Expliquez sans le faire comment on peut vérifier cette hypothèse.
    
\paragraph{Solution}

%????
le rapport de risque (hazard ratio) entre les individus est constant.Cela signifie que l'ampleur relative du risque entre les individus ne change pas au fil du temps.
    
Nous pouvons utiliser les diagrammes des résidus de Schoenfeld, 
qui montrent les résidus de Schoenfeld 
(une mesure de l'impact des variables sur le risque au fil du temps) 
en fonction du temps. Si l'hypothèse de risque proportionnel se vérifie, 
ces points ne présenteraient aucune tendance systématique dans le temps, 
c'est-à-dire qu'ils seraient distribués de manière aléatoire autour d'une ligne horizontale.

\end{CJK*}
\end{document}
