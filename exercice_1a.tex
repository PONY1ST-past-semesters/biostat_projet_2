\documentclass[../main.tex]{subfiles}

\begin{document}
\begin{CJK*}{UTF8}{gbsn}

\section*{Exercice 1a}
Décrivez brièvement les variables du jeu de données 
\texttt{diabetic} dans la bibliothèque \texttt{survival} en R.
Quelle était une question de recherche menant à cette collecte de données ?

\paragraph{Solution}

\begin{itemize}
  \item \textbf{ID} : Il est utilisé pour distinguer chaque participant dans l'ensemble de données.
  \item \textbf{laser} : Il s'agit du type de traitement laser reçu. $1=\text{xenon}, 2=\text{argon}$.
  \item \textbf{age} : Il s'agit de l'âge auquel le diabète a été diagnostiqué chez le patient.
  \item \textbf{eye} : Il s'agit d'un facteur avec des niveaux de gauche et de droit.
  \item \textbf{trt} : Il s'agit du groupe de traitement. $0=\text{contrôle}, 1=\text{laser}$.
  \item \textbf{risk} : Il s'agit de classer les participants dans les groupes de risques différents.
  Les valeurs varient de $6$ à $12$ où $6$ représente le groupe avec risque le plus petit.
  \item \textbf{time} : Il s'agit du nombre de jours du début de la recherche à la cécité 
  ou à la dernière observation.
  \item \textbf{status} : Il s'agit d'une variable binaire 
  pour indiquer si une perte de vision s'est produite au cours de la période d'étude, 
  où $0$ signifie qu'on ne perd pas de vision et $1$ pour le contraire.
\end{itemize}

Ces données proviennent d'une étude d'analyse de 
la survie de patients atteints de rétinopathie diabétique à haut risque, 
conçue pour évaluer l'efficacité du traitement au laser dans le ralentissement de la progression de la cécité. ////

\end{CJK*}
\end{document}
